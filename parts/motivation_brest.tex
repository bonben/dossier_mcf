\section{Lettre de motivation}

L'Institut Mines-Télécom Atlantique est le produit de la fusion de l'école des Mines de Nantes et de l'école Télécom Bretagne. 
Sa mission est de former des ingénieurs généralistes dans les thématiques du numérique, de l'énergie et de l'environnement.
Il propose également des formations pour l'obtention du grades de Master.
Les enseignants-chercheurs de l'IMT Atlantique sont répartis dans des départements.
Le poste dont fait l'objet cette candidature est proposé par le département \'Electronique.
Les enseignants de ce département interviennent dans les enseignements de tronc commun de 1\textsuperscript{ère} année dans la thématique \og électronique et capteurs \fg{}.
Ils interviendront également dans deux futures thématiques d'approfondissement de 2\textsuperscript{ème} et 3\textsuperscript{ème} année : Systèmes Embarqués et Hétérogènes (SEH) et Conception d'Objets Communicants (CoOC).
% Manque transition
Les membres du département \'Electronique font partie de l'équipe \og Interaction Algorithme-Silicium \fg du laboratoire CNRS Lab-Sticc.
En tant qu'anciens membres de Télécom Bretagne, l'expertise d'une majeure partie des checheurs de l'équipe de ce département est appliquée au domaine des communications numériques. 
La contribution la plus marquante de l'équipe du département électronique est la propositions d'une classe de codes correcteurs d'erreurs nommés turbo codes par Claude Berrou. 
Cette invention a fortement orientés les travaux des membres de l'équipe du département vers des propositions algorithmiques et architecturales dans le domaine de la correction d'erreurs et des processus itératifs dans les chaînes de communications numériques.

Comme il est détaillé dans les projets d'intégration, mon expérience en enseignement et en recherche me semble en adéquation avec le département électronique.
En effet, j'ai assuré la plupart de mes cours dans les filères \'Electronique et Systèmes \'Electronique Embarqués (filière par alternance) de l'Enseirb-Matmeca à Talence. Le contenu de formation en électronique est assez proche de celui de l'IMT Atlantique.
Du point de vue de la recherche, le sujet de ma thèse de doctorat portait sur l'implémentation d'algorithmes de décodage de codes polaires, sur des architectures programmables. Les codes polaires sont une famille de codes correcteurs d'erreurs définie il y a une décennie par Erdal Ar\i{}kan, qui suscitent beaucoup d'intérêt comme en atteste leur intégration dans la norme 5G. Le lien avec les thématiques de recherche de l'équipe IAS est donc évident. En attestent par exemple les thèses en cours concernant l'implémentation matérielle de l'algorithme de décodage par annulation successive à liste. Outre l'algorithme considéré, les cibles choisies pour leur implémentations sont un autre lien entre mes travaux et ceux de l'équipe IAS. Deux de mes contributions concernent des architectures ASIP qui sont des architectures sur lesquelles le Pr. Amer Baghdadi a beaucoup travaillé.

Par ailleurs, mes travaux sur les architectures ASIP ont d'ailleurs été mené à Polytechnique Montréal. En effet, j'ai réalisé ma thèse en cotutelle entre l'Université de Bordeaux et Polytechnique Montréal. Cela m'a permis de construire un réseau d'étudiants et de professeurs qui pourront être des interlocuteurs pertinents pour la création de futurs partenariats, donnant ainsi à ma candidature la dimension internationale requise.

Si les points de rencontre sont nombreux, je pense que certaines de mes compétences sont complémentaires avec celles des membres de l'équipe IAS. J'éprouve un grand intérêt dans l'implémentation logicielle des algorithmes de décodage de codes correcteurs d'erreurs. De telles implémentations peuvent avoir de nombreuses applications, comme leur utilisation dans des radios logicielles ou bien dans le cadre de simulations intensives. Une partie des mes travaux de recherche a concerné des implémentations logicielles de certains algorithmes de décodage sur des processeurs généralistes d'architecture ARM ou x86. De telles implémentations me paraissent très pertinentes. Tout d'abord, elles suscitent l'intérêt des industriels du secteur car elles pourraient permettre des réseaux plus flexibles et adaptables. D'autre part, il s'agit d'un outil formidable pour tester de nouveaux algorithmes ou de nouveaux systèmes. Pour ce faire, toutes les possibilités de parallélisations de ces cibles doivent être exploitées : parallélisme de données (instructions SIMD), parallélisme d'instruction (exécution multifils), calcul distribué (exécution multin\oe{}uds).

Ces techniques avancées de programmation pourraient par ailleurs constituer le c\oe{}ur d'une UE dans la future thématique d'approfondissement Systèmes Embarqués et Hétérogènes. De telles interactions entre travaux de recherche et d'enseignement sont l'essence même du métier d'enseignant-chercheur que j'espère avoir la chance d'exercer.
