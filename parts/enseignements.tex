
\begin{center}
\begin{tabular}{|c|c|c|c|}
\hline
\textbf{\'Etablissement} & \textbf{Ann\'ee scolaire} & \textbf{Vol. horaire \'eq. TD} & \textbf{Section} \\ \hline \hline
{ENSEIRB-MATMECA} & {2015 - 2016} & {45 h} & 61 \\ \hline
{ENSEIRB-MATMECA} & {2018 - 2019} & {192 h} & 61 \\
\hline
\multicolumn{2}{c|}{} & 237 h \'eq. TD \\
\cline{3-3}
\end{tabular}
\end{center}
%\vspace{1em}


\subsubsection{Projet de conception en \'electronique}
\label{subsubsec:loto}
\begin{description}\parskip 0pt
\item[Responsable :] Mathieu L\'eonardon
\item[Niveau :] $1^{\mbox{\`ere}}$ ann\'ee ENSEIRB - Fili\`ere SEE- \'Equivalent Licence 3
\item[Volume :] 2 x 25 HETD
\item[Contenu :]  
La premi\`ere partie de ce module est un cours sur le flot de conception FPGA.
Les machines d'\'etats sont \'egalement pr\'esent\'ees et quelques exercices sont propos\'es.
S'en suit un projet de conception d'une architecture num\'erique. Les objectifs sont :
        \begin{itemize}
        \item concevoir une architecture mat\'erielle,
        \item acqu\'erir des comp\'etence en VHDL,
        \item d\'evelopper un esprit synth\'etique pour la r\'edaction du rapport,
        \item acqu\'erir des comp\'etences en pr\'esentation de projet avec une soutenance.
    \end{itemize}
    
Le projet est la r\'ealisation d'un Loto sur une carte Nexys 4. L'utilisateur final doit pouvoir tirer 6 nombres al\'eatoirement, sans retirage. Ces nombres sont affich\'es sur des afficheurs 7 segments.
\item[Mots-cl\'es :] FPGA, VHDL, Electronique num\'erique 
\item[Participation personnelle :] Cours int\'egr\'e, encadrement du projet. 
\item[Ressources p\'edagogiques cr\'e\'ees :] \url{https://bit.ly/2HLfaXt}
\end{description}


\subsubsection{Architecture reconfigurable}
\label{subsubsec:reconf}
\begin{description}\parskip 0pt
\item[Responsable :]  Mathieu L\'eonardon
%\item[Effectif :] Environ 15 \'el\`eves
\item[Niveau :] $2^{\mbox{\`eme}}$ ann\'ee ENSEIRB-MATMECA - Fili\`ere SEE - \'Equivalent Master 1
\item[Volume :] 1 x 20 HETD
\item[Contenu :] Conception avanc�e sur les circuits FPGA. Objectifs :
    \begin{itemize}
        \item d\'efinir les architectures reconfigurables,
        \item comprendre leur structure interne,
        \item comprendre leur fonctionnement,
        \item mise en oeuvre de techniques avanc\'ees,
        \item acqu\'erir des comp\'etences en pr\'esentation de projet avec une petite soutenance.
    \end{itemize}
\item[Mots-cl\'es :] FPGA, VHDL, Electronique num\'erique 
\item[Participation personnelle :] Cours int\'egr\'e, encadrement d'un projet.
\item[Ressources p\'edagogiques cr\'e\'ees :] \url{https://bit.ly/2U4Ua4o }
\end{description}

\subsubsection{\'Electronique Num\'erique}
\label{subsubsec:rsi}
\begin{description}\parskip 0pt
\item[Responsable :] Mathieu L\'eonardon
%\item[Effectif :] Environ 15 \'el\`eves
\item[Niveau :] $1^{\mbox{\`ere}}$ ann\'ee ENSEIRB-MATMECA - Fili\`ere RSI- \'Equivalent Licence 3
\item[Volume :] 1 x 25 HETD
\item[Contenu :] Ce module pr\'esente les notions de bases de l'\'electronique num\'erique : la num\'eration, l'alg\`ebre de Boole, la logique combinatoire ainsi que la logique s\'equentielle. Architecture et fonctionnement d'une machine \`a \'etats finis. Bases de technologie des circuits imprim\'es.
\item[Mots-cl\'es :] Num\'eration, Machine \`a \'etats finis, technologie CMOS, Electronique num\'erique 
\item[Participation personnelle :] Cours int\'egr\'e - TD
\item[Ressources p\'edagogiques cr\'e\'ees :] \url{https://bit.ly/2U0hgsV}
\end{description}


\subsubsection{Logique combinatoire et logique s\'equentielle}
\label{subsubsec:en1}
\begin{description}\parskip 0pt
\item[Responsable :] Christophe J\'ego
%\item[Effectif :] Environ 15 \'el\`eves
\item[Niveau :] $1^{\mbox{\`ere}}$ ann\'ee ENSEIRB-MATMECA - Fili\`ere \'Electronique- \'Equivalent Licence 3
\item[Volume :] 1 x 32 HETD
\item[Contenu :] ~

\begin{itemize}
\item Les fonctions \'el\'ementaires combinatoires et s\'equentielles utilis\'ees dans les circuits num\'eriques,
\item la mod\'elisation des fonctions num\'eriques \`a l'aide du langage VHDL.
\end{itemize}
\`A l'issue du cours, l'\'etudiant doit \^etre capable :
\begin{itemize} 
\item de d\'ecrire une fonction combinatoire et la repr\'esenter par un circuit num\'erique,
\item de d\'ecrire et synth\'etiser un compteur, une machine \`a \'etats,
\item de rep\'erer le chemin critique d'une fonction logique complexe et de calculer sa fr\'equence maximale de fonctionnement.
\end{itemize}
Six s\'eances de travaux dirig\'es de quatre heures compl\`etes le cours. Chaque s\'eance se d\'ecompose en deux parties. 
Des th\`emes sont trait\'es dans une premi\`ere partie. 
Dans une seconde partie, les syst\`emes num\'eriques d\'efinis sont d\'ecrits dans le langage de description mat\'eriel VHDL. 
Cette approche permet d'initier progressivement les \'etudiants \`a ce langage. 
\item[Mots-cl\'es :] Electronique num\'erique, VHDL, FPGA
\item[Participation personnelle :] Encadrement des TP et du projet
\end{description}

\noindent \subsubsection{Projet micro-processeur}
\begin{description}\parskip 0pt
\item[Responsable :] Val\'ery Lebret
\item[Niveau :] $1^{\mbox{\`ere}}$ ann\'ee ENSEIRB-MATMECA - Fili\`ere \'Electronique- \'Equivalent Licence 3
\item[Volume :] 1 x 36 HETD
\item[Contenu :] Cet enseignement a pour objectif la programmation de microcontr\^oleurs PIC de MICROCHIP, choisis pour leur facilit\'e de mise en oeuvre li\'e \`a leur faible complexit\'e. 
Apr\`es une pr\'esentation de cette famille de microcontr\^oleurs et de leurs sp\'ecificit\'es, l'activit\'e commence par l'\'ecriture de programmes simples en langage assembleur visant \`a illustrer le fonctionnement du microcontr\^oleur (codage et ex\'ecution des instructions, acc\`es aux registres, gestions des ressources internes et des entr\'ees/sorties...). 
Une carte d'application int\'egrant un PIC16F84 sert de support, le d\'eveloppement logiciel se faisant gr\^ace \`a la chaine d'outils int\'egr\'es MPLABX qui dispose notamment d'un simulateur. 
La programmation s'effectue ensuite en langage C avec pour finalit\'e la mise en oeuvre d'un projet (par exemple une horloge \`a quartz sur afficheur LCD) au moyen de la carte de d\'eveloppement PICDEM2 comportant une cible PIC16F877 (plus de ressources internes, possibilit\'e de faire du d\'ebogage). L'accent est mis sur les limitations rencontr\'ees sur les syst\`emes embarqu\'es lors de la programmation en langage C (espace m\'emoire r\'eduit, puissance de calcul limit\'ee, ..) ainsi que sur la gestion des interruptions.
\item[Mots-cl\'es :] Programmation microcontr\^oleur, langage assembleur
\item[Participation personnelle :] Encadrement des TP
\end{description}


\noindent \subsubsection{Architecture des ordinateurs}
\begin{description}\parskip 0pt
\item[Responsable :] J\'er\'emie Crenne
\item[Niveau :] $2^{\mbox{\`eme}}$ ann\'ee ENSEIRB-MATMECA - Fili\`ere \'Electronique- \'Equivalent Master 1
\item[Volume :] 1 x 16 HETD
\item[Contenu :] Cet enseignement a pour but de renforcer les connaissances en abordant des techniques plus avanc\'ees relatives aux processeurs et aux m\'emoires. Ce cours est articul\'e autour du livre de J.L. Hennessy et A. Patterson "Computer Architecture, a quantitative approach". Sont abord\'es les architectures RISC et CISC, les architecture pipeline, les al\'eas de donn\'ees et de contr\^ole. Le cas du processeur MIPS est \'etudi\'e en d\'etail pour donner aux \'etudiants un exemple concret d'architecture de processeur.
La finalit\'e de ce cours est de permettre aux \'etudiants de comprendre les syst\`emes multi/many-coeurs les plus sophistiqu\'es.
\item[Mots-cl\'es :] Architecture des ordinateurs, architectures pipelines, multifil
\item[Participation personnelle :] Encadrement des TD
\end{description}

\noindent \subsubsection{Projet micro-informatique}
\begin{description}\parskip 0pt
\item[Responsable :] Yannick Bornat
\item[Niveau :] $2^{\mbox{\`eme}}$ ann\'ee ENSEIRB-MATMECA - Fili\`ere \'Electronique- \'Equivalent Master 1
\item[Volume :] 1 x 42 HETD
\item[Contenu :] 
L'ensemble de TPs s'effectue sur microcontr\^oleur AT91SAM7X256. Ce microcontr\^oleur poss\`ede un coeur ARM7, de nombreux p\'eriph\'eriques ainsi qu'un syst\`eme d'interruptions vectoris\'ees et param\'etrable.
L'objectif de l'enseignement est d'utiliser les diff\'erentes ressources mat\'erielles pour concevoir un syst\`eme d'exploitation minimaliste couvrant les besoins sp\'ecifiques des microcontr\^oleurs pour des t\^aches temps r\'eel.
Les notions abord\'ees couvrent les aspects temps-r\'eel, communication, int\'egrit\'e des donn\'ees.
Les \'etudiants sont plac\'es dans une situation o\`u leur seul source de documentation est constitu\'ee par les documents techniques constructeur en anglais.

\item[Mots-cl\'es :] Programmation microcontr\^oleur, langage C, r\'eseaux d'interruptions
\item[Participation personnelle :] Encadrement des TP et du projet
\end{description}

\noindent \subsubsection{Programmation objet. Langage C++}
\begin{description}\parskip 0pt
\item[Responsable :] Bertrand Le Gal
\item[Niveau :] $2^{\mbox{\`eme}}$ ann\'ee ENSEIRB-MATMECA - Fili\`ere \'Electronique- \'Equivalent Master 1
\item[Volume :] 1 x 15 HETD
\item[Contenu :] 
Cet enseignement vise \`a apporter aux \'etudiants les base de la programmation orient\'ee objets. Les concepts g\'en\'eraux de la programmation orient\'ee objets sont introduits en cours. Le langage C++ est utilis\'e afin d'illustrer les concepts manipul\'es. L'ensemble de ces notions sont mises \`a profit dans un projet afin d'illustrer de mani\`ere pratique l'interet de cette approche de programmation.
\item[Mots-cl\'es :] Programmation orient\'ee objet, C++
\item[Participation personnelle :] Encadrement des TP
\end{description}

\clearpage
%###################
